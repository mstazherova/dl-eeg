\documentclass{amsart}

\usepackage[round]{natbib}
\usepackage{graphicx}
\usepackage[notref,notcite]{showkeys}
\usepackage{url}

% in case of supplementary material
% \usepackage{xr}
% \externaldocument{lcfit_supp}


% general
\newcommand{\EE}{\mathbb{E}}

% arxivness
\newcommand{\arxiv}[1]{#1}
\newcommand{\notarxiv}[1]{}
\newcommand{\eat}[1]{}

% theorems, etc
\newtheorem{lemma}{Lemma}
\newtheorem{prop}{Proposition}
\newtheorem{theorem}{Theorem}
\newtheorem{problem}{Problem}
\newtheorem{defn}{Definition}
\newtheorem{obs}{Observation}
\newtheorem{alg}{Algorithm}

\newcommand{\FIGshape}{\
\begin{figure}
\begin{center}
  \arxiv{\includegraphics[height=9cm]{theta_shape.pdf}}
\end{center}
\caption{\
  As $\theta$ goes to zero, the $g_\theta$ converge pointwise to $g$, which is 1 on the interior of the unit interval and 0 on the boundaries.
}
\label{FIGshape}
\end{figure}
}
\newcommand{\refFIGshape}{1}


\title[Shorttitle]{Longtitle}
\author{Author}
\date{\today}


\begin{document}

\begin{abstract}
Start, stop.
\end{abstract}

\maketitle

\section{Introduction}
Introduction content

\section{Data}
Data content

\section{Methods}
Methods content

\section{Experiments}
Experiments content

\section{Discussion}
Discussion content

\bibliography{lcfittex}
\bibliographystyle{alphaplainnat}

% Figure legends for submission
\notarxiv{
\section{Figure Legends}
\FIGshape
\renewcommand{\thefigure}{S\arabic{figure}} % Figure S1, etc.
\setcounter{figure}{0} % Reset counter.
}


\bibliographystyle{plain}
\bibliography{skeleton.bib}
\end{document}